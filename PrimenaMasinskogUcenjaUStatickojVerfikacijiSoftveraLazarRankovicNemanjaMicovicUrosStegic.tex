% !TEX encoding = UTF-8 Unicode

\documentclass[a4paper]{article}

\usepackage{color}
\usepackage{url}
\usepackage[T2A]{fontenc} % enable Cyrillic fonts
\usepackage[utf8]{inputenc} % make weird characters work
\usepackage{graphicx}

%\usepackage[english,serbian]{babel}
\usepackage[english,serbianc]{babel} %ukljuciti babel sa ovim opcijama, umesto gornjim, ukoliko se koristi cirilica

\usepackage[unicode]{hyperref}
\hypersetup{colorlinks,citecolor=green,filecolor=green,linkcolor=blue,urlcolor=blue}

%\newtheorem{primer}{Пример}[section] %ćirilični primer
\newtheorem{primer}{Primer}[section]

\begin{document}

%TODO smisliti naslov
\title{Примена машинског учења у статичкој верификацији софтвера\\ \small{Семинарски рад у оквиру курса\\Методологија стручног и научног рада\\ Математички факултет}}

\author{Лазар Ранковић, Немања Мићовић, Урош Стегић\\ lazar.rankovic@outlook.com, nmicovic@outlook.com, mi10287@alas.matf.bg.ac.rs}
\date{}
\maketitle

\abstract{
%U ovom tekstu je ukratko prikazana osnovna forma seminarskog rada. Obratite pažnju da je pored ove .pdf datoteke, u prilogu i odgovarajuća .tex datoteka, kao i .bib datoteka korišćena za generisanje literature. Na prvoj strani seminarskog rada su naslov, apstrakt i sadržaj, i to sve mora da stane na prvu stranu! Kako bi Vaš seminarski zadovoljio standarde i očekivanja, koristite uputstva i materijale sa predavanja na temu pisanja seminarskih radova. Ovo je samo šablon koji se odnosi na fizički izgled seminarskog rada (šablon koji \emph{morate} da ispoštujete!) kao i par tehničkih pomoćnih uputstava. Molim Vas da kada budete predavali seminarski rad, imenujete datoteke tako da sadrže temu seminarskog rada, kao i imena i prezimena članova grupe (ili samo temu i prezimena, ukoliko je sa imenima predugačko). Predaja seminarskih radova biće isključivo preko web forme, a NE slanjem mejla.
Пошто ћемо абстракт писати на крају, онда док радимо да искористимо
ово за интерне потребе. До сада прегледани радови:\\
\begin{itemize}
	\item{Finding latent code errors via machine learning over program executions
		\cite{Brun04findinglatent}}
	\item{Learning Invariants using Decision Trees
		\cite{KrishnaPW15}}
	\item{ICE: A Robust Framework for Learning Invariants
		\cite{Garg2014}}
	\item{Interpolants as Classifiers
		\cite{Sharma_interpolantsas}}
	\item{Advanced Verification Techniques Based on learning
		\cite{Jain}}
	\item{A Survey of Static Program Analysis Techniques
		\cite{survey}}
	\item{A Survey of Automated Techniques for Formal Software Verification
		\cite{dkw2008}}
	\item{Regresiona verifikacija softvera korišćenjem sistema LAV
		\cite{milena}}
\end{itemize}

\tableofcontents

\newpage

\section{Uvod}
Увод ћемо пред крај писати.

\section{Верификација софтвера}

Питања се извлаче из Милениног доктората или пронаћи неки рад чисто о верификацији.\\
Одговара се на питања:
\begin{itemize}
\item Шта је верификација
\item Зашто је важна
\item Уопштено како се ради
\end{itemize}

По пар реченица за ове пасусе:
\paragraph{Динамичка верификација}
Ово не радимо у раду, само треба кратак опис. Није довољно прецизна (реф.)
\paragraph{Статичка верификација}
Овиме се бавимо и поента је што је она прецизна 


\section{Технике статичке верификације}
Општа прича, које су врсте (набројати макар три :Р) и за сваки тип по један параграф.
Постоје:
\begin{itemize}
\item Апстрактна интерпретација
\item Симболичко израчунавање
\item Проверавање ограничених модела (енг. Bounded model checking)
\end{itemize}
Литература:
A Survey of Static Program Analysis Techniques \cite{survey}\\
A Survey of Automated Techniques for Formal Software Verification \cite{dkw2008}\\
Миленин докторат (ово је још и најбоље)

\section{Машинско учење}
Одговара се на питања:
\begin{itemize}
\item Шта је МЛ
\item Зашто је важно
\item Уопштено како се ради
\end{itemize}
Описа надгледаног и ненадгледаног учења, јако кратко.

\section{Веза верификације и МЛ}
Шта су проблеми верификације, због чега се кочи, који су изазови. Неки начини
решавања (паралелизација, хеуристике...)\\
Описати који су делови верификације где лепо легне МЛ (увод у следеће поглавље).

\section{Неке примене техника машинског учења у статичкој верификацији}
Ово је есенција. Одабирају се проблеми из претходног поглавља и показује се
како се решава. Прво иде неки уводни део, онда из литературе се покупе те технике
и таксативно се наводе (принцип проблем-решење).

\section{Закључак}
Овде машти на вољу.. :)


\addcontentsline{toc}{section}{Literatura}
\appendix
\bibliography{seminarski} 
\bibliographystyle{plain}


\end{document}
